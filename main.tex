\documentclass[french]{report}

\usepackage{amsmath}
\usepackage{amssymb}
\usepackage{amsthm}
\usepackage{babel}

\title{La fonction zêta de Riemann}
\author{Amaury Martiny}
\date{\today}

\newtheorem{theorem}{Théorème}[section]
\newtheorem{definition}[theorem]{Définition}
\newtheorem{proposition}[theorem]{Proposition}
\newtheorem{corollary}[theorem]{Corollaire}
\newtheorem{lemma}[theorem]{Lemme}

\begin{document}
\maketitle

\begin{abstract}
  This is the paper's abstract \ldots
  \end{abstract}

\tableofcontents{}

%%%%%%%%%%%%%%%%%%%%%%%%%%%%%%%%%%%%%%%%%%%%%%%%%%%%%%%%%%%%%%%%%%%%%%%%%%%%%%%%
%%%%%%%%%%%%%%%%%%%%%%%%%%%%%%%%%%%%%%%%%%%%%%%%%%%%%%%%%%%%%%%%%%%%%%%%%%%%%%%%
%%%%%%%%%%%%%%%%%%%%%%%%%%%%%%%%%%%%%%%%%%%%%%%%%%%%%%%%%%%%%%%%%%%%%%%%%%%%%%%%
\chapter{Les nombres premiers}
%%%%%%%%%%%%%%%%%%%%%%%%%%%%%%%%%%%%%%%%%%%%%%%%%%%%%%%%%%%%%%%%%%%%%%%%%%%%%%%%
%%%%%%%%%%%%%%%%%%%%%%%%%%%%%%%%%%%%%%%%%%%%%%%%%%%%%%%%%%%%%%%%%%%%%%%%%%%%%%%%
%%%%%%%%%%%%%%%%%%%%%%%%%%%%%%%%%%%%%%%%%%%%%%%%%%%%%%%%%%%%%%%%%%%%%%%%%%%%%%%%

%%%%%%%%%%%%%%%%%%%%%%%%%%%%%%%%%%%%%%%%%%%%%%%%%%%%%%%%%%%%%%%%%%%%%%%%%%%%%%%%
%%%%%%%%%%%%%%%%%%%%%%%%%%%%%%%%%%%%%%%%%%%%%%%%%%%%%%%%%%%%%%%%%%%%%%%%%%%%%%%%
\section{Un peu d'histoire}
%%%%%%%%%%%%%%%%%%%%%%%%%%%%%%%%%%%%%%%%%%%%%%%%%%%%%%%%%%%%%%%%%%%%%%%%%%%%%%%%
%%%%%%%%%%%%%%%%%%%%%%%%%%%%%%%%%%%%%%%%%%%%%%%%%%%%%%%%%%%%%%%%%%%%%%%%%%%%%%%%

%%%%%%%%%%%%%%%%%%%%%%%%%%%%%%%%%%%%%%%%%%%%%%%%%%%%%%%%%%%%%%%%%%%%%%%%%%%%%%%%
%%%%%%%%%%%%%%%%%%%%%%%%%%%%%%%%%%%%%%%%%%%%%%%%%%%%%%%%%%%%%%%%%%%%%%%%%%%%%%%%
\section{Quelques premiers résultats}
%%%%%%%%%%%%%%%%%%%%%%%%%%%%%%%%%%%%%%%%%%%%%%%%%%%%%%%%%%%%%%%%%%%%%%%%%%%%%%%%
%%%%%%%%%%%%%%%%%%%%%%%%%%%%%%%%%%%%%%%%%%%%%%%%%%%%%%%%%%%%%%%%%%%%%%%%%%%%%%%%

Mertens, $\sum_p\frac{\ln p}{p}$

%%%%%%%%%%%%%%%%%%%%%%%%%%%%%%%%%%%%%%%%%%%%%%%%%%%%%%%%%%%%%%%%%%%%%%%%%%%%%%%%
%%%%%%%%%%%%%%%%%%%%%%%%%%%%%%%%%%%%%%%%%%%%%%%%%%%%%%%%%%%%%%%%%%%%%%%%%%%%%%%%
\section{Tchebychev et ses fonctions}
%%%%%%%%%%%%%%%%%%%%%%%%%%%%%%%%%%%%%%%%%%%%%%%%%%%%%%%%%%%%%%%%%%%%%%%%%%%%%%%%
%%%%%%%%%%%%%%%%%%%%%%%%%%%%%%%%%%%%%%%%%%%%%%%%%%%%%%%%%%%%%%%%%%%%%%%%%%%%%%%%

Nous introduisons ici les fonctions de Tchebychev, car elles vont nous suivre dans toute la suite de ce rapport.

\begin{definition}[Fonctions de Tchebychev] Pour $x\in \mathbb{R}$,
  \[ \theta(x) = \sum_{p \le x}\log p \]
\end{definition}

%%%%%%%%%%%%%%%%%%%%%%%%%%%%%%%%%%%%%%%%%%%%%%%%%%%%%%%%%%%%%%%%%%%%%%%%%%%%%%%%
%%%%%%%%%%%%%%%%%%%%%%%%%%%%%%%%%%%%%%%%%%%%%%%%%%%%%%%%%%%%%%%%%%%%%%%%%%%%%%%%
\section{Les méthodes du crible}
%%%%%%%%%%%%%%%%%%%%%%%%%%%%%%%%%%%%%%%%%%%%%%%%%%%%%%%%%%%%%%%%%%%%%%%%%%%%%%%%
%%%%%%%%%%%%%%%%%%%%%%%%%%%%%%%%%%%%%%%%%%%%%%%%%%%%%%%%%%%%%%%%%%%%%%%%%%%%%%%%

%%%%%%%%%%%%%%%%%%%%%%%%%%%%%%%%%%%%%%%%%%%%%%%%%%%%%%%%%%%%%%%%%%%%%%%%%%%%%%%%
%%%%%%%%%%%%%%%%%%%%%%%%%%%%%%%%%%%%%%%%%%%%%%%%%%%%%%%%%%%%%%%%%%%%%%%%%%%%%%%%
%%%%%%%%%%%%%%%%%%%%%%%%%%%%%%%%%%%%%%%%%%%%%%%%%%%%%%%%%%%%%%%%%%%%%%%%%%%%%%%%
\chapter{Notre boîte à outils pour l'analyse}
%%%%%%%%%%%%%%%%%%%%%%%%%%%%%%%%%%%%%%%%%%%%%%%%%%%%%%%%%%%%%%%%%%%%%%%%%%%%%%%%
%%%%%%%%%%%%%%%%%%%%%%%%%%%%%%%%%%%%%%%%%%%%%%%%%%%%%%%%%%%%%%%%%%%%%%%%%%%%%%%%
%%%%%%%%%%%%%%%%%%%%%%%%%%%%%%%%%%%%%%%%%%%%%%%%%%%%%%%%%%%%%%%%%%%%%%%%%%%%%%%%

TODO Série de Dirichlet

%%%%%%%%%%%%%%%%%%%%%%%%%%%%%%%%%%%%%%%%%%%%%%%%%%%%%%%%%%%%%%%%%%%%%%%%%%%%%%%%
%%%%%%%%%%%%%%%%%%%%%%%%%%%%%%%%%%%%%%%%%%%%%%%%%%%%%%%%%%%%%%%%%%%%%%%%%%%%%%%%
\section{Outils d'analyse réelle}
%%%%%%%%%%%%%%%%%%%%%%%%%%%%%%%%%%%%%%%%%%%%%%%%%%%%%%%%%%%%%%%%%%%%%%%%%%%%%%%%
%%%%%%%%%%%%%%%%%%%%%%%%%%%%%%%%%%%%%%%%%%%%%%%%%%%%%%%%%%%%%%%%%%%%%%%%%%%%%%%%

%%%%%%%%%%%%%%%%%%%%%%%%%%%%%%%%%%%%%%%%%%%%%%%%%%%%%%%%%%%%%%%%%%%%%%%%%%%%%%%%
\subsection{La sommation d'Abel}
%%%%%%%%%%%%%%%%%%%%%%%%%%%%%%%%%%%%%%%%%%%%%%%%%%%%%%%%%%%%%%%%%%%%%%%%%%%%%%%%

Nous allons utiliser à plusieurs reprises la formule d'Abel. Elle fournit une comparaison assez précise entre somme et intégrale.

\begin{lemma}\label{lem:formule-abel}
  Soit $(a_n)_{n\in\mathbb{N}}$ une suite numérique réelle, et $A(x):=\sum_{n\leq x}a_n$. On se donne également une fonction $f$ de classe $C^1$. Alors :
  \[ \forall y>x\geq0,\quad
  \sum_{x<n\leq y} a_n f(n)
  = A(y)f(y) - A(x)f(x)
  - \int_x^yA(t)f'(t)\mathrm{d}t
  \]
\end{lemma}

\begin{proof}
  On remarque que $A$ est en escalier, donc pour tout $n\geq0$,
  \[ \int_n^{n+1}A(t)f'(t)\mathrm{d}t
  = A(n)\int_n^{n+1}f'(t)\mathrm{d}t
  = A(n)(f(n+1)-f(n)).
  \]

  Posons $M=|x|$ et $N=|y|$, alors
  \begin{align*}
    \int_x^yA(t)f'(t)\mathrm{d}t
    &= \sum_{n=M}^{N-1}\int_n^{n+1}A(t)f'(t)\mathrm{d}t \\
    &= \sum_{n=M}^{N-1}A(n)(f(n+1)-f(n)) \\
    &= \sum_{n=M+1}^Nf(n)(A(n)-A(n-1))-A(N)f(N)+A(M)f(M) \\
    &= \sum_{n=M+1}^Nf(n)a_n-A(N)f(N)+A(M)f(M)
  \end{align*}

  D'où la formule quand $x$ et $y$ sont entiers. Pour la formule générale on observe que :
  \[ -\int_{|x|}^xA(t)f'(t)\mathrm{d}t = A(|x|)(f(x)-f(|x|)) = A(x)(f(x)-f(|x|)). \]
  
\end{proof}

%%%%%%%%%%%%%%%%%%%%%%%%%%%%%%%%%%%%%%%%%%%%%%%%%%%%%%%%%%%%%%%%%%%%%%%%%%%%%%%%
\subsection{Formule d'Euler-Maclaurin}
%%%%%%%%%%%%%%%%%%%%%%%%%%%%%%%%%%%%%%%%%%%%%%%%%%%%%%%%%%%%%%%%%%%%%%%%%%%%%%%%

\begin{theorem}\label{eq:euler-maclaurin}
  Pour tout entier $k\geq0$ et toute fonction $f$ de classe $C^r$ sur $[a,b]$, $a,b\in\mathbb{Z}$, on a
  \begin{align*}
    \sum_{n=a}^b f(n) = \int_a^b f(t)\mathrm{d}t \quad + \quad \frac{f(a) + f(b)}{2} \quad &+ \quad \sum_{k=2}^r\frac{b_{k}}{k!}(f^{(k-1)}(b) - f^{(k-1)(a)}) \\
    &+ \quad \frac{(-1)^{r+1}}{r!}\int_a^b B_r(t)f^{(r)}(t)\mathrm{d}t
  \end{align*}

  Les $b_n$ sont les nombres de Bernoulli, et les $B_n$ sont les polynômes de Bernoulli, définis sur $[0,1]$ par la récurrence classique, et ensuite prolongés par 1-périodicité.
\end{theorem}

%%%%%%%%%%%%%%%%%%%%%%%%%%%%%%%%%%%%%%%%%%%%%%%%%%%%%%%%%%%%%%%%%%%%%%%%%%%%%%%%
%%%%%%%%%%%%%%%%%%%%%%%%%%%%%%%%%%%%%%%%%%%%%%%%%%%%%%%%%%%%%%%%%%%%%%%%%%%%%%%%
\section{Outils d'analyse complexe}
%%%%%%%%%%%%%%%%%%%%%%%%%%%%%%%%%%%%%%%%%%%%%%%%%%%%%%%%%%%%%%%%%%%%%%%%%%%%%%%%
%%%%%%%%%%%%%%%%%%%%%%%%%%%%%%%%%%%%%%%%%%%%%%%%%%%%%%%%%%%%%%%%%%%%%%%%%%%%%%%%

%%%%%%%%%%%%%%%%%%%%%%%%%%%%%%%%%%%%%%%%%%%%%%%%%%%%%%%%%%%%%%%%%%%%%%%%%%%%%%%%
\subsection{La fonction $\Gamma$ d'Euler}
%%%%%%%%%%%%%%%%%%%%%%%%%%%%%%%%%%%%%%%%%%%%%%%%%%%%%%%%%%%%%%%%%%%%%%%%%%%%%%%%

\begin{theorem}\label{thm:stirling-complexe}
  Pour tout $s\in\mathbb{C}\backslash\mathbb{R}$,
  \[
    \log\Gamma(s)
    = (s-\frac{1}{2})\log s
    - s
    + \frac{1}{2}\ln(2\pi)
    - \int_0^\infty B_1(t)\frac{\mathrm{d}t}{s+t}
  \]
\end{theorem}

%%%%%%%%%%%%%%%%%%%%%%%%%%%%%%%%%%%%%%%%%%%%%%%%%%%%%%%%%%%%%%%%%%%%%%%%%%%%%%%%
\subsection{Séries de Dirichlet}
%%%%%%%%%%%%%%%%%%%%%%%%%%%%%%%%%%%%%%%%%%%%%%%%%%%%%%%%%%%%%%%%%%%%%%%%%%%%%%%%

Le titre de ce rapport contient "la fonction $\zeta$" de Riemann, mais avant de la définir, nous allons d'abord définir ce qu'est une série de Dirichlet.

\begin{definition}\label{def:serie-dirichlet}
  Soit $f$ une fonction arithmétique. Alors la série de Dirichlet associée à $f$ est la série (formelle) :
  \[ F(s) = \sum_{n=1}^\infty\frac{f(n)}{n^s}. \]
\end{definition}

Nous n'allons pas démontrer le résultat suivant, voir TODO
% http://mural.maynoothuniversity.ie/4470/1/finaldraftmsc.pdf p24 
% WWL CHEN.Distribution of Prime Numbers, A Series of Lectures, (1981)
pour un démonstration

\begin{theorem}
  Soit $F$ une série de Dirichlet, et supposons qu'il existe $s\in\mathbb{C}$ telle que $F(s)$ converge. Alors il existe trois réels :
  \[ -\infty\leq\sigma_c\leq\sigma_u\leq\sigma_a\leq+\infty \]
  tels que :
  \begin{itemize}
    \item $F$ converge pour tout $s$ vérifiant $\sigma>\sigma_c$.
  \end{itemize}
\end{theorem}

%%%%%%%%%%%%%%%%%%%%%%%%%%%%%%%%%%%%%%%%%%%%%%%%%%%%%%%%%%%%%%%%%%%%%%%%%%%%%%%%
\subsection{Formule de Perron}
%%%%%%%%%%%%%%%%%%%%%%%%%%%%%%%%%%%%%%%%%%%%%%%%%%%%%%%%%%%%%%%%%%%%%%%%%%%%%%%%

Nous établissons dans cette section la formule de Perron, reliant une série de Dirichlet à sa fonction sommatoire normalisée.

Dans toute cette section, nous allons noter

\[ F(s) =\sum_{n\geq1}\frac{a_n}{n^s}\]

une série de Dirichlet. On note :
\begin{itemize}
  \item $\sigma_c$ son abscisse de convergence simple,
  \item $\sigma_a$ son abscisse de convergence absolue.
\end{itemize}

Prolongeons la suite $(a_n)_n$ en une fonction sur $\mathbb{R}$ en posant
\[
  a_x=
  \begin{cases}
    a_n & \text{si}\,n\in\mathbb{N} \\
    0  & \text{sinon}
  \end{cases}
\]
\begin{definition}[Fonction sommatoire normalisée]
  Nous introduisons :
  \[ A^*(x) = \sum_{n<x}a_n + \frac{1}{2}a_x \]
\end{definition}

\begin{theorem}[Formule de Perron]
  Soit $\kappa>\max(0,\sigma_c)$. On a :
  \[ A^*(x) = \frac{1}{2\pi i}\int_{\kappa-i\infty}^{\kappa+i\infty}F(s)\frac{x^s}{s}\mathrm{d}s \]
\end{theorem}

où l'intégrale est
\begin{itemize}
  \item convergeante en valeur principale lorsque $x\in\mathbb{N}$
  \item semi-convergeante lors $x\in\mathbb{R}\backslash\mathbb{N}$.
\end{itemize}

\begin{proof}
  TODO
\end{proof}


Dans la pratique, nous n'utilison pas la formule de Perron telle quelle, mais sous une forme "effective".

\begin{theorem}[Première formule de Perron effective]
  Pour $\kappa>\max(0,\sigma_c)$, $T\geq1$, $x\geq1$,
  \[ A(x) = \frac{1}{2\pi i}\int_{\kappa-i\infty}^{\kappa+i\infty}F(s)\frac{x^s}{s}\mathrm{d}s + O\left(x^k\sum_{n\geq1}\frac{|a_n|}{n^\kappa(1+T|\ln\frac{x}{n}|)}\right) \]
\end{theorem}

\begin{proof}
  TODO
\end{proof}

\begin{theorem}[Seconde formule de Perron effective]\label{eq:perron-2-effective}
  On suppose que :
  \begin{itemize}
    \item l'abscisse de convergence absolue de $F(s)$ $\sigma_a$ est finie,
    \item il existe un nombre réel $\alpha\geq0$ tel que
    \[ \forall\sigma\in]\sigma_a,\sigma_a+1],\quad\sum_{n\geq1}\frac{|a_n|}{n^{-\sigma}}=O((\sigma-\sigma_a)^{-\alpha}) \]
    \item il existe une fonction B positive et croissante telle que
    \[ \forall n\geq1,\quad|a_n|\leq B(n) \]
  \end{itemize}
  Alors, pour $x\geq2$, $T\geq2$, $\sigma\leq\sigma_a$, $\kappa:=\sigma_a-\sigma+1/\ln x$,
  \[ \sum_{n\leq x}\frac{a_n}{n^s}=\frac{1}{2\pi i}\int_{\kappa-iT}^{\kappa+iT}F(s+\omega)\frac{x^\omega}{\omega}\mathrm{d}\omega + O\left(x^{\sigma_a-\sigma}\frac{\ln^\alpha x}{T}+\frac{B(2x)}{x^\sigma}\left(1+x\frac{\ln T}{T}\right)\right) \]
\end{theorem}

\begin{proof}
  TODO
\end{proof}

%%%%%%%%%%%%%%%%%%%%%%%%%%%%%%%%%%%%%%%%%%%%%%%%%%%%%%%%%%%%%%%%%%%%%%%%%%%%%%%%
%%%%%%%%%%%%%%%%%%%%%%%%%%%%%%%%%%%%%%%%%%%%%%%%%%%%%%%%%%%%%%%%%%%%%%%%%%%%%%%%
%%%%%%%%%%%%%%%%%%%%%%%%%%%%%%%%%%%%%%%%%%%%%%%%%%%%%%%%%%%%%%%%%%%%%%%%%%%%%%%%
\chapter{La fonction $\zeta$ de Riemann}
%%%%%%%%%%%%%%%%%%%%%%%%%%%%%%%%%%%%%%%%%%%%%%%%%%%%%%%%%%%%%%%%%%%%%%%%%%%%%%%%
%%%%%%%%%%%%%%%%%%%%%%%%%%%%%%%%%%%%%%%%%%%%%%%%%%%%%%%%%%%%%%%%%%%%%%%%%%%%%%%%
%%%%%%%%%%%%%%%%%%%%%%%%%%%%%%%%%%%%%%%%%%%%%%%%%%%%%%%%%%%%%%%%%%%%%%%%%%%%%%%%

%%%%%%%%%%%%%%%%%%%%%%%%%%%%%%%%%%%%%%%%%%%%%%%%%%%%%%%%%%%%%%%%%%%%%%%%%%%%%%%%
%%%%%%%%%%%%%%%%%%%%%%%%%%%%%%%%%%%%%%%%%%%%%%%%%%%%%%%%%%%%%%%%%%%%%%%%%%%%%%%%
\section{Lien avec les nombres premiers}
%%%%%%%%%%%%%%%%%%%%%%%%%%%%%%%%%%%%%%%%%%%%%%%%%%%%%%%%%%%%%%%%%%%%%%%%%%%%%%%%
%%%%%%%%%%%%%%%%%%%%%%%%%%%%%%%%%%%%%%%%%%%%%%%%%%%%%%%%%%%%%%%%%%%%%%%%%%%%%%%%

La fonction que nous appelons aujourd'hui fonction $\zeta$ de Riemann a en réalité été introduite par Euler au XVIIIème siècle. Il a défini, pour tout $x\in\mathbb{R}, x>1$, la fonction

\[ \zeta(x) = \sum_{n=1}^{\infty}\frac{1}{n^x} \]

La somme de droite est clairement convergente, donc $\zeta(x)$ est bien défini. Euler démontra par la suite le résultat suivant, qui définit un lien entre les nombres premiers et l'analyse.

\begin{theorem}[Produit eulérien]\label{thm:produit-eulerien-reel}
  \[ \zeta(x) = \prod_p\frac{1}{1-p^{-x}}\quad(\sigma>1) \]
  Plus précisément, le produit infini signifie :
  \[ \lim_{P\to+\infty}\prod_{p\,\mathrm{premier}\\p\leq P}\frac{1}{1-p^{-x}}=\zeta(x)\quad(\sigma>1) \]
\end{theorem}

\begin{proof}
  Pour tout nombre premier $p$, comme $1/p<1$, on peut écrire la somme d'une suite géométrique :
  \[ \frac{1}{1-p^{-x}}=\sum_{k=0}^{+\infty}\frac{1}{p^{kx}}. \]

  En faisant le produit de cette égalité pour tous les nombres premiers $p_1, ..., p_r$ inférieurs à un certain $T$, on a :
  
  \begin{align}
    \prod_{p\leq T}\frac{1}{1-p^{-x}}
    &= \prod_{p\leq T}\sum_{k=0}^{+\infty}\frac{1}{p^{kx}}  \\
    &= \sum_{m_1,...,m_r\geq1}\frac{1}{(p_1^{m_1}...p_r^{m_r})^x} \label{eq:produit-euler-part1} \\
    &= \sum_{n\in\mathbb{N}_T}n^{-x} \label{eq:produit-euler-part2}
  \end{align}

  où l'on a noté $\mathbb{N}_T$ l'ensemble des entiers naturels dont tous les facteurs premiers sont inférieurs ou égaux à $T$.

  (\ref{eq:produit-euler-part1}) est jusitifiée par le fait que tous les termes sont positifs, donc on peut intervertir les sommes. (\ref{eq:produit-euler-part2}) vient de la décomposition unique en facteurs premiers.
  \\

  Ainsi,
  \[ \left|\sum_{n=1}^{\infty}\frac{1}{n^x}-\prod_{p\leq T}\frac{1}{1-p^{-x}}\right|
  = \left|\sum_{n\notin\mathbb{N}_T}\frac{1}{n^x}\right|
  \leq \sum_{n\notin\mathbb{N}_T}\frac{1}{n^x}
  \leq \sum_{n>T}\frac{1}{n^x}
  \]

  La dernière somme est le reste d'une série convergente, donc tend vers 0, ce qui prouve à la fois la convergence du produit et la formule d'Euler.
\end{proof}

A ce stade, nous commençons à nous convaincre du rôle important de la fonction $\zeta$ dans l'étude des nombres premiers. Une analyse plus approfondie de cette fonction va nous aider grandement, c'est ce que nous allons faire tout de suite.

%%%%%%%%%%%%%%%%%%%%%%%%%%%%%%%%%%%%%%%%%%%%%%%%%%%%%%%%%%%%%%%%%%%%%%%%%%%%%%%%
%%%%%%%%%%%%%%%%%%%%%%%%%%%%%%%%%%%%%%%%%%%%%%%%%%%%%%%%%%%%%%%%%%%%%%%%%%%%%%%%
\section{Quelques propriétés de $\zeta$}
%%%%%%%%%%%%%%%%%%%%%%%%%%%%%%%%%%%%%%%%%%%%%%%%%%%%%%%%%%%%%%%%%%%%%%%%%%%%%%%%
%%%%%%%%%%%%%%%%%%%%%%%%%%%%%%%%%%%%%%%%%%%%%%%%%%%%%%%%%%%%%%%%%%%%%%%%%%%%%%%%

Avant d'étudier ses propriétés, commençons par définir officiellement $\zeta$. L'idée de Riemann, dans son TODO, a été de partir de la définition d'Euler, et de considérer $\zeta$ comme fonction d'une variable complexe :

\begin{definition}[Fonction $\zeta$ de Riemann]\label{def:zeta-definition}
  On définit, pour tout $s$ complexe tel que $\sigma > 1$,
  \[ \zeta(s) = \sum_{n=1}^{\infty}\frac{1}{n^s} \]
\end{definition}

Remarquons que $\zeta$ est la série de Dirichlet (voir la définition \ref{def:serie-dirichlet}) associée à la fonction constante $1$.

\begin{proposition}
  Pour $\sigma > 1$, la série $\zeta(s)$ est absolument convergente.
\end{proposition}

\begin{proof}
  C'est évident car $|\frac{1}{n^s}| = \frac{1}{n^{\sigma}}$, qui est le terme général d'une série convergente.
\end{proof}

Cette proposition montre que la fonction $\zeta$ est bien définie sur le demi-plan $\sigma > 1$.
\\

Cela va sans dire, mais cela ira encore mieux en le disant :

\begin{theorem}[Produit eulérien, variable complexe]\label{thm:produit-eulerien-complexe}
  \[ \zeta(s) = \prod_p\frac{1}{1-p^{-s}}\quad(\sigma>1)\]
\end{theorem}

\begin{proof}
  La démonstration est exactement la même que dans le cas réel : voir la proposition \ref{thm:produit-eulerien-reel}. Il suffit de notre que $|\frac{1}{n^s}|=\frac{1}{n^\sigma}$.
\end{proof}

Il en découle immédiatement cette 1ère propriété intéressante :

\begin{proposition}\label{eq:zeta-non-nul-produit-eulerien}
  \[ \zeta(s) \neq 0 \quad(\sigma > 1)\]
\end{proposition}

\begin{proof}
  Soit $p$ premier fixé. L'inégalité triangulaire donne $|1-p^{-s}|\leq1+p^{-\sigma}$. Par conséquent,
  \[\log(|1-p^{-s}|)\leq\log(1+p^{-\sigma})\leq p^{-\sigma} \]
  où les inégalités sont données respectivement par la croissance et par la concavité du logarithme.

  Ceci entraîne la convergence de la série $\sum_p\log(|1-p^{-s}|)$, pour tout $s$ avec $\sigma>1$. Notons $L_s$ sa limite:
  \[ \log\left(\prod_p|1-p^{-s}|\right) = L_s, \]
  et ainsi
  \[ \left|\prod_p\frac{1}{1-p^{-s}}\right| = \mathrm{e}^{-L_s} > 0, \]
  Le terme de gauche est exactement $|\zeta(s)|$ par le produit eulérien du théorème \ref{thm:produit-eulerien-complexe}.

\end{proof}

\begin{proposition}\label{eq:zeta-majoration-facile}
  \[ |\zeta(s)|\leq\frac{\sigma}{\sigma-1}\quad(\sigma>1) \]
\end{proposition}

\begin{proof}
  TODO
\end{proof}

\begin{proposition}
  $\zeta$ est holomorphe sur le demi-plan $\sigma > 1$.
\end{proposition}

\begin{proof}
  Soit $K$ un compact du demi-plan $\sigma > 1$, alors $K$ est inclus dans un $\{ s\in\mathbb{C} \mid \sigma \geq a \}$ pour un certain réel $a > 1$. Mais alors en définissant $f_n: s \mapsto 1 / n^s = \mathrm{e}^{-s\ln n}$, la fonction $f_n$ est holomorphe sur $\sigma > 1$, et sur $K$, $\Vert{f_n}\Vert_\infty \leq 1 / n^a$, qui est le terme général d'une série convergente.
  
  La série de fonctions $\sum f_n$ converge vers $\zeta$, normalement (donc uniformément) sur $K$. Par suite $\zeta$ est holomorphe sur le demi-plan $\sigma > 1$.
\end{proof}

%%%%%%%%%%%%%%%%%%%%%%%%%%%%%%%%%%%%%%%%%%%%%%%%%%%%%%%%%%%%%%%%%%%%%%%%%%%%%%%%
\subsection{Expression de $\log\zeta$ et fonction de van Mangoldt}
%%%%%%%%%%%%%%%%%%%%%%%%%%%%%%%%%%%%%%%%%%%%%%%%%%%%%%%%%%%%%%%%%%%%%%%%%%%%%%%%

$\zeta(s)$ est réel pour $s$ réel, et est strictement positif. Le logarithme de ce nombre existe et est réel. Il est donc naturel de choisir, parmi tous les $\log\zeta$ possible, de choisir celle qui prolonge $\ln\zeta$ sur l'axe réel.

\begin{definition}[Fonction de van Mangoldt]
\[
  \Lambda(n)=
  \begin{cases}
    \ln p & \text{s'il existe $m\geq1$ et $p$ tels que $n=p^m$} \\
    0 & \text{sinon}
  \end{cases}
\]
\end{definition}

\begin{proposition}\label{prop:log-zeta}
  On peut construuire une détermination holomorphe de $\log\zeta$ pour $\sigma>1$ en posant
  \[
    \log\zeta(s)
    = \sum_p\sum_{\nu\geq1}\frac{1}{\nu p^{\nu s}}
    = \sum_{n=2}^\infty\frac{\Lambda(n)}{n^s\ln n}\quad(\sigma>1).
  \]

  Alors pour $x$ réel,
  \[ \log\zeta(x) = \ln\zeta(x). \]
\end{proposition}

\begin{proof}
  D'une part, pour $s=\sigma$ réel et en partant du produit eulérien \ref{thm:produit-eulerien-complexe}, on a
  \[ \ln\zeta(\sigma) = -\sum_p\ln\left(1-\frac{1}{p^\sigma}\right). \]

  D'autre part, on développe le logarithme en série entière $\ln(1-u)=u+\frac{u^2}{2}+...+\frac{u^k}{k}+...$, ce qui est possible puisque $p\geq2$ et $\sigma>1$, et on peut donc définir
  \[ D(s) = \sum_p\sum_{\nu\geq1}\frac{1}{\nu p^{\nu s}}. \]
  
  La série $D$ est normalement convergente sur tout compact du demi-plan $\sigma>1$, donc y définit une fonction holomorphe.

  Sur la demi-droite réelle $s=\sigma>1$,
  \[ \sum_{\nu\geq1}\frac{1}{\nu p^{\nu s}} = -\ln\left(1-\frac{1}{p^\sigma}\right), \]

  donc $\mathrm{e}^{D(s)}=\zeta(s)$.
  \\
  
  $\mathrm{e}^D$ et $\zeta$ sont donc deux fonctions holomorphes du demi-plan ouvert $\sigma>1$, et coïncident sur la demi-droite réelle $s=\sigma>1$. Par unicité du prolongement, elles coïncident sur tout le demi-plan $\sigma>1$.

  En dérivant l'égalité $\mathrm{e}^D$ et $\zeta$, on obtient immédiatement
  \[ D'(s)=\frac{\zeta'(s)}{\zeta(s)}, \]
  
  et en dérivant terme à terme la série définissant $D$, 
  \[ D'(s) = -\sum_p\sum_{\nu\geq1}\frac{\ln p}{p^{\nu s}}, \]

  de sorte que l'on a, pour tout $\sigma>1$,
  \[
    D'(s)
    = \frac{\zeta'(s)}{\zeta(s)}
    = -\sum_p\sum_{\nu\geq1}\frac{\ln p}{p^{\nu s}}
    = -\sum_{n=2}^\infty\frac{\Lambda(n)}{n^s}.
  \]

  En intégrant terme à terme, la série $D$ se réecrit alors:
  \[ D(s) = \sum_{n=2}^\infty\frac{\Lambda(n)}{n^s\ln n} \]

  ce qui prouve la proposition.
\end{proof}

Nous allons extraire une partie de la preuve ci-dessus, en la mettre dans le corollaire suivant :

\begin{corollary}\label{cor:zeta-zeta-prime-van-mangoldt}
  On a l'égalité
  \[ -\frac{\zeta'(s)}{\zeta(s)} = \sum_{n\geq1}\frac{\Lambda(n)}{n^s}\quad(\sigma>1) \]
\end{corollary}

\begin{proposition}\label{prop:zeta-sur-zeta-prime-1-sur-holomorphe}
  La fonction suivant est analytique au voisinage de 1 :
  \[ F(s) = - \frac{\zeta'(s)}{\zeta(s)} - \frac{1}{s-1} \]
\end{proposition}

Cette proposition découle immédiatement du lemme suivant :

\begin{lemma}
  Soit $f$ méromorphe avec un pôle d'order $k$ en $s=\alpha$. Alors $\frac{f}{f'}$ a un pôle d'order 1 en $s=\alpha$, de résidu $-k$.
\end{lemma}

\begin{proof}
  On peut écrire
  \[ f(s) = \frac{g(s)}{(s-\alpha)^k},\]
  avec $g$ holomorphe au voisinage de $\alpha$ et $g(\alpha)\neq0$.
  \\

  Donc pour tout $s$ au voisinage de $\alpha$,
  \[
    f'(s)
    = \frac{g'(s)}{(s-\alpha)^k}-\frac{kg(s)}{(s-\alpha)^{k+1}}
    = \frac{g(s)}{(s-\alpha)^k}\left(\frac{g'(s)}{g(s)}-\frac{k}{s-\alpha}\right)
  \]

  Donc
  \[
    \frac{f'(s)}{f(s)} = \frac{-k}{s-\alpha}+\frac{g'(s)}{g(s)}.
  \]

  Comme $\frac{g'(s)}{g(s)}$ est holomorphe au voisinage de $\alpha$, ceci prouve le lemme.
\end{proof}

\begin{proof}[Preuve de la proposition \ref{prop:zeta-sur-zeta-prime-1-sur-holomorphe}]
 $-\frac{\zeta'}{\zeta}$ a un pôle d'ordre 1 en 1, $\frac{1}{s-1}$ a un pôle d'ordre 1 en 1. Leur différence est donc holomorphe au voisiange de 1.
\end{proof}

\begin{corollary}\label{cor:zeta-sur-zeta-prime-o}
  \[ \left|\frac{\zeta'(\sigma)}{\zeta(\sigma)}\right| = O_{\sigma\to1^+}\left(\frac{1}{\sigma-1}\right) \]
\end{corollary}

%%%%%%%%%%%%%%%%%%%%%%%%%%%%%%%%%%%%%%%%%%%%%%%%%%%%%%%%%%%%%%%%%%%%%%%%%%%%%%%%
\subsection{Expression intégrale}
%%%%%%%%%%%%%%%%%%%%%%%%%%%%%%%%%%%%%%%%%%%%%%%%%%%%%%%%%%%%%%%%%%%%%%%%%%%%%%%%

\begin{proposition}
  Pour tout complexe $\sigma > 1$,
  \[ \Gamma(s)\zeta(s) = \int_0^\infty\frac{t^{s-1}}{\mathrm{e}^t-1}\mathrm{d}t \]
\end{proposition}

\begin{proof}
  On part de la formule TODO
  \[ \Gamma(s)n^{-s} = \int_0^\infty t^{s-1}\mathrm{e}^{-nt}\mathrm{d}t\quad (\sigma > 0) \]

  En sommant pour $n\geq1$, il vient pour $\sigma>1$
  \[ \Gamma(s)\zeta(s) = \sum_{n=1}^\infty\int_0^\infty t^{s-1}\mathrm{e}^{-nt}\mathrm{d}t = \int_0^\infty \frac{t^{s-1}}{\mathrm{e}^t - 1}\mathrm{d}t \]
  Remarquons que la série numérique $\sum\int_0^\infty |t^{s-1}|\mathrm{e}^{-nt}\mathrm{d}t = \sum\int_0^\infty t^{\sigma-1}\mathrm{e}^{-nt}\mathrm{d}t$ converge, donc l'interversion somme/intégrale est justifiée.
\end{proof}

%%%%%%%%%%%%%%%%%%%%%%%%%%%%%%%%%%%%%%%%%%%%%%%%%%%%%%%%%%%%%%%%%%%%%%%%%%%%%%%%
%%%%%%%%%%%%%%%%%%%%%%%%%%%%%%%%%%%%%%%%%%%%%%%%%%%%%%%%%%%%%%%%%%%%%%%%%%%%%%%%
\section{Equation fonctionnelle}
%%%%%%%%%%%%%%%%%%%%%%%%%%%%%%%%%%%%%%%%%%%%%%%%%%%%%%%%%%%%%%%%%%%%%%%%%%%%%%%%
%%%%%%%%%%%%%%%%%%%%%%%%%%%%%%%%%%%%%%%%%%%%%%%%%%%%%%%%%%%%%%%%%%%%%%%%%%%%%%%%

\begin{theorem}[Equation fonctionnelle, Riemann]\label{thm:equation-fonctionnelle}
  \[ \zeta(s) = 2^s\pi^{s-1}\sin\left(\frac{\pi s}{2}\right)\Gamma(1-s)\zeta(1-s)\quad (s\in\mathbb{C}\backslash\{0,1\}) \]
\end{theorem}

\begin{proof}
  TODO
  % https://www.uni-ulm.de/fileadmin/website_uni_ulm/mawi.inst.020/abschlussarbeiten/BA_wiesel.pdf
\end{proof}

%%%%%%%%%%%%%%%%%%%%%%%%%%%%%%%%%%%%%%%%%%%%%%%%%%%%%%%%%%%%%%%%%%%%%%%%%%%%%%%%
%%%%%%%%%%%%%%%%%%%%%%%%%%%%%%%%%%%%%%%%%%%%%%%%%%%%%%%%%%%%%%%%%%%%%%%%%%%%%%%%
\section{Prolongement et propriétés sur la bande critique}
%%%%%%%%%%%%%%%%%%%%%%%%%%%%%%%%%%%%%%%%%%%%%%%%%%%%%%%%%%%%%%%%%%%%%%%%%%%%%%%%
%%%%%%%%%%%%%%%%%%%%%%%%%%%%%%%%%%%%%%%%%%%%%%%%%%%%%%%%%%%%%%%%%%%%%%%%%%%%%%%%

La définition de $\zeta$ que nous avons donnée en \ref{def:zeta-definition} est valable pour $\sigma>1$. Nous allons essayer, dans cette section et dans la suivante, d'essayer de prolonger $\zeta$ sur d'autres parties du plan complexe. \\

Nous commençons par la prolonger jusqu'au demi-plan $\sigma>0$. Le fermé de la région prolongée, $0\leq\sigma\leq1$, s'appelle la $bande\,critique$. Au fait, de nombreux résultats arithmétiques reliant $\zeta$ et les nombres premiers font appel aux propriétés de $\zeta$ sur cette bande critique.

\begin{proposition} On a, pour $s\neq0$ et $\sigma>0$,
  \[ \zeta(s) = \frac{s}{s-1}-s\int_1^\infty\frac{\{u\}}{u^{1+s}}\mathrm{d}u \]
\end{proposition}

\begin{proof}
  Sommation d'abel, TODO faire proprement
\end{proof}


\begin{lemma}\label{lem:zeta-somme-partielle}
  On a, pour $N\geq0$ et $\sigma>0$,
  \[ \zeta(s) = \sum_{n=1}^N\frac{1}{n^s} + \frac{N^{1-s}}{s-1}-s\int_N^{\infty}\frac{\{u\}}{u^{1+s}}\mathrm{d}u \]
\end{lemma}

\begin{proposition}\label{prop:majoration-zeta-et-zeta-prime}
  Pour tout réel $A>0$, sur la région du plan définie par $\{s\in\mathrm{C}\,|\,t\geq2,\,\sigma>1-\frac{A}{\ln t}\}$, on a :
  \[ \zeta(s) = O(\ln t) \]
  \[ \zeta'(s) = O(\ln^2 t) \]
  où la constante dans le $O$ dépend de $A$.
\end{proposition}

\begin{proof}
  Les égalités $|\zeta(2)|\leq\zeta(2)$ et $|\zeta'(2)|\leq\zeta'(2)$ sont triviales, donc les deux majorations sont vérifiées pour $\sigma\geq2$.

  On peut donc supposer maintenant que $\sigma<2$ et $t\geq2$, que l'on fixe tous les deux. On a alors (la première inégalité vient de l'inégalité triangulaire) :
  \[ |s|\leq\sigma+t<2+t\leq 2t \]
  et
  \[ |s-1|\geq t. \]
  En reportant tout cela dans le lemme \ref{lem:zeta-somme-partielle}, on obtient, pour tout $N\geq1$ :
  \begin{equation}\label{eq:zeta-abs-majoree}
    |\zeta(s)|\leq\sum_{n=1}^N\frac{1}{n^\sigma}
    + \frac{N^{1-\sigma}}{t}
    + 2t\int_n^\infty\frac{1}{u^{1+\sigma}}\mathrm{d}u
  \end{equation}
 
  En prenant TODO $N=|t|$, on a
  \begin{itemize}
    \item $N\leq t<N+1,$
    \item $\forall n\leq N,\,\ln n\leq\ln t.$
  \end{itemize}

  Par ailleurs, l'hypothèse $1-\sigma<A/\ln t$ entraîne
  \[ \frac{1}{n^\sigma}
  = \frac{1}{n}n^{1-\sigma}
  = \frac{1}{n}\mathrm{e}^{(1-\sigma)\ln n}
  < \frac{1}{n}\mathrm{e}^{A\frac{\ln n}{\ln t}}
  \leq \frac{1}{n}\mathrm{e}^A
  = O\left(\frac{1}{n}\right)
  \]

  On peut donc maintenant évaluer terme à terme l'équation \ref{eq:zeta-abs-majoree}.

  \[ \sum_{n=1}^N\frac{1}{n^\sigma}
  = O\left(\sum_{n=1}^N\frac{1}{n}\right)
  = O(\ln N)
  = O(\ln t)
  \]

  \[ \frac{N^{1-\sigma}}{t}
  = \frac{N}{t}\frac{1}{N^\sigma}
  = O\left(\frac{1}{N}\right)
  = O(1)
  \]

  \[ 2t\int_n^\infty\frac{1}{u^{1+\sigma}}\mathrm{d}u
  = \frac{2t}{\sigma N^\sigma}
  = O\left(\frac{2t}{N}\right)
  = O(1)
  \]

  En remplaçant dans \ref{eq:zeta-abs-majoree},
  \[ |\zeta(s)|=O(\ln t) \]
\end{proof}

Par symétrie par rapport à l'axe $t=0$, on a également :

\begin{corollary}\label{cor:majoration-zeta-et-zeta-prime}
  Pour tout réel $A>0$, sur la région du plan définie par $\{s\in\mathrm{C}\,|\,\,\sigma>1-\frac{A}{\ln (2+|t|)}\}$, on a :
  \[ \zeta(s) = O(\ln (2+|t|)) \]
  \[ \zeta'(s) = O(\ln^2 (2+|t|)) \]
  où la constante dans le $O$ ne dépend que de $A$.
\end{corollary}

%%%%%%%%%%%%%%%%%%%%%%%%%%%%%%%%%%%%%%%%%%%%%%%%%%%%%%%%%%%%%%%%%%%%%%%%%%%%%%%%
%%%%%%%%%%%%%%%%%%%%%%%%%%%%%%%%%%%%%%%%%%%%%%%%%%%%%%%%%%%%%%%%%%%%%%%%%%%%%%%%
\section{Prolongement à $\mathbb{C}\backslash\{1\}$}
%%%%%%%%%%%%%%%%%%%%%%%%%%%%%%%%%%%%%%%%%%%%%%%%%%%%%%%%%%%%%%%%%%%%%%%%%%%%%%%%
%%%%%%%%%%%%%%%%%%%%%%%%%%%%%%%%%%%%%%%%%%%%%%%%%%%%%%%%%%%%%%%%%%%%%%%%%%%%%%%%

\begin{theorem}
  La fonction $\zeta$ admet un unique prolongement en une fonction méromorphe sur $\mathbb{C}$ ayant un unique pôle en $s=1$ de résidu 1.
\end{theorem}

Une fois ce théorème démontré, nous allons encore noter $\zeta$ cet unique prolongement. Nous allons donner plusieurs démonstrations de ce théorème.

\subsection{Par la formule d'Euler-Maclaurin}

\begin{proof}
Fixons $s$ tel que $\sigma>1$, et appliquons la formule d'Euler-Maclaurin \ref{eq:euler-maclaurin} à l'ordre $r\geq1$ sur l'intervalle $[1,N]$ à la fonction $f:t\mapsto t^{-s}$, de classe $C^\infty$ sur $[1,N]$ :

\begin{align*}
  \sum_{n=1}^N n^{-s} = \frac{1-N^{1-s}}{s-1} + \frac{1+N^{-s}}{2} &+ \sum_{k=2}^r B_k\frac{s(s+1)...(s+k-2)}{k!}(1-N^{-s-k+1}) \\
  &- R_{r,N}(s)
\end{align*}
où l'on a défini le reste $ R_{r,N}(s)$ par

\[  R_{r,N}(s) = \frac{s(s+1)...(s+r-1)}{r!}\int_1^N B_r(t)t^{-s-r}\mathrm{d}t \]

Comme les $B_r$ sont périodiques et polynomiaux sur $[0,1[$, ils sont bornés. Le terme à l'intérieur de l'intégrale de $R_{r,N}(s)$ est donc $O(t^{-s-r})$, qui intégrable sur $[1, +\infty]$. En faisant tender $N$ vers l'infini, on obtient alors

\[ \zeta(s) = \frac{1}{s-1} + F_r(s) \]

où l'on a noté

\[ F_r(s) = \frac{1}{2} + \sum_{k=2}^r B_k\frac{s(s+1)...(s+k-2)}{k!} - \frac{s(s+1)...(s+r-1)}{r!}\int_1^\infty B_r(t)t^{-s-r}\mathrm{d}t. \]

Montrons que $F_r$ est holomorphe sur $\Omega_r=\{ s\in\mathbb{C}\,|\,\sigma > 1-r\}$. Il suffit que montrer que $G_r$ l'est, où

\[ G_r(s) = \int_1^\infty B_r(t)t^{-s-r}\mathrm{d}t.\]

On remarque que, à $t$ fixé, la fonction à l'intérieur $s\mapsto B_r(t)t^{-s-r}$ l'est. Soit $K$ un compact de $\Omega_r$, on peut fixer un $\delta>0$ tel que $K\subset\{ s\in\mathbb{C}\,|\,\sigma > 1-r+\delta\}$. Sur ce compact,

\[ \sup_{s\in K}\left|\frac{B_r(t)}{t^{s+r}}\right| = O\left(\frac{1}{t^{1+\delta}}\right) \]

ce qui assure, par régularité des intégrales à paramètre, que $G_r$, et par suite $F_r$ est holomorphe sur $\Omega_r$.

On peut ainsi définir une fonction entière $F$ par

\[ F(s) = F_r(s) \quad\text{si}\,s\in\Omega_r. \]

$F$ est bien définie car si $1\leq q\leq r$, $F_q(s)=F_r(s)=\zeta(s)-\frac{1}{s-1}$, donc $F_q$ et $F_r$ sont holomorphes et coincident sur $\Omega_q$ connexe.

On obtient finalement que $s\mapsto\frac{1}{s-1}+F(s)$ est une fonction méromorphe avec un unique pôle simple en 1 de résidu 1 qui prolonge la fonction $\zeta$ de Riemann.

L'unicité est triviale par prolongement analytique.
\end{proof}

%%%%%%%%%%%%%%%%%%%%%%%%%%%%%%%%%%%%%%%%%%%%%%%%%%%%%%%%%%%%%%%%%%%%%%%%%%%%%%%%
\subsection{Par un contour de Hankel}
%%%%%%%%%%%%%%%%%%%%%%%%%%%%%%%%%%%%%%%%%%%%%%%%%%%%%%%%%%%%%%%%%%%%%%%%%%%%%%%%

% https://papyrus.bib.umontreal.ca/xmlui/bitstream/handle/1866/14610/Samson_Jean-Philippe_2003_memoire.pdf?sequence=1&isAllowed=y page39

%%%%%%%%%%%%%%%%%%%%%%%%%%%%%%%%%%%%%%%%%%%%%%%%%%%%%%%%%%%%%%%%%%%%%%%%%%%%%%%%
\subsection{Par l'équation fonctionnelle}
%%%%%%%%%%%%%%%%%%%%%%%%%%%%%%%%%%%%%%%%%%%%%%%%%%%%%%%%%%%%%%%%%%%%%%%%%%%%%%%%

Prolongement sur $[0,1]$ par la sommation d'abel, sur $\sigma<0$ par l'équation fonctionnelle.

%%%%%%%%%%%%%%%%%%%%%%%%%%%%%%%%%%%%%%%%%%%%%%%%%%%%%%%%%%%%%%%%%%%%%%%%%%%%%%%%
%%%%%%%%%%%%%%%%%%%%%%%%%%%%%%%%%%%%%%%%%%%%%%%%%%%%%%%%%%%%%%%%%%%%%%%%%%%%%%%%
\section{Développement en produit d'Hadamard}
%%%%%%%%%%%%%%%%%%%%%%%%%%%%%%%%%%%%%%%%%%%%%%%%%%%%%%%%%%%%%%%%%%%%%%%%%%%%%%%%
%%%%%%%%%%%%%%%%%%%%%%%%%%%%%%%%%%%%%%%%%%%%%%%%%%%%%%%%%%%%%%%%%%%%%%%%%%%%%%%%

%%%%%%%%%%%%%%%%%%%%%%%%%%%%%%%%%%%%%%%%%%%%%%%%%%%%%%%%%%%%%%%%%%%%%%%%%%%%%%%%
\subsection{Lemmes d'analyse complexe}
%%%%%%%%%%%%%%%%%%%%%%%%%%%%%%%%%%%%%%%%%%%%%%%%%%%%%%%%%%%%%%%%%%%%%%%%%%%%%%%%

%%%%%%%%%%%%%%%%%%%%%%%%%%%%%%%%%%%%%%%%%%%%%%%%%%%%%%%%%%%%%%%%%%%%%%%%%%%%%%%%
\subsection{Produit d'Hadamard}
%%%%%%%%%%%%%%%%%%%%%%%%%%%%%%%%%%%%%%%%%%%%%%%%%%%%%%%%%%%%%%%%%%%%%%%%%%%%%%%%

\begin{definition}
  On définit, pour $s\in\mathbb{C}\backslash\{0,1\}$,
  \[ \xi(s) := s(s-1)\pi^{-s/2}\Gamma(s/2)\zeta(s) \]
\end{definition}

\begin{proposition}
  La fonction $\xi$ vérifie l'équation fonctionelle
  \[ \xi(s) = \xi(1-s) \]
\end{proposition}

\begin{proof}
  C'est juste un reformulation du théorème \ref{thm:equation-fonctionnelle}.
\end{proof}

\begin{theorem}\label{thm:produit-hadamard}
  Il existe des constantes $a$ et $b$ telles que
  \[ \xi(s)=\mathrm{e}^{as}\prod_{\rho}\left(1-\frac{s}{\rho}\right)\mathrm{e}^{s/\rho} \]
  \[ \zeta(s)=\frac{\mathrm{e}^{bs}}{2(s-1)}\Gamma(s/2+1)^{-1}\prod_{\rho}\left(1-\frac{s}{\rho}\right)\mathrm{e}^{s/\rho} \]
\end{theorem}

\begin{proof}
  TODO
\end{proof}

%%%%%%%%%%%%%%%%%%%%%%%%%%%%%%%%%%%%%%%%%%%%%%%%%%%%%%%%%%%%%%%%%%%%%%%%%%%%%%%%
%%%%%%%%%%%%%%%%%%%%%%%%%%%%%%%%%%%%%%%%%%%%%%%%%%%%%%%%%%%%%%%%%%%%%%%%%%%%%%%%
\section{Localisation des zéros}
%%%%%%%%%%%%%%%%%%%%%%%%%%%%%%%%%%%%%%%%%%%%%%%%%%%%%%%%%%%%%%%%%%%%%%%%%%%%%%%%
%%%%%%%%%%%%%%%%%%%%%%%%%%%%%%%%%%%%%%%%%%%%%%%%%%%%%%%%%%%%%%%%%%%%%%%%%%%%%%%%

Nous avons vu à la proposition \ref{eq:zeta-non-nul-produit-eulerien} que $\zeta$ ne s'annule pas pour $\sigma>1$. Nous allons montrer que cette propriété est vraie sur le demi-plan fermé $\sigma\geq1$.

\begin{theorem}[Mertens]\label{thm:mertens-positif}
  Soit $F(s) =\sum_{n\geq1}\frac{a_n}{n^s}$ une série de Dirichlet à coefficients positits ou nuls. Notions $\sigma_c$ sont abscisse de convergence. Alors:
  \[ 3F(\sigma)+4\Re(F(\sigma+it))+\Re(F(\sigma+2it))\geq0 \quad(\sigma>\sigma_c) \]
\end{theorem}

\begin{proof}
  Posons $\forall\theta\in\mathbb{R}, V(\theta) := 3+4\cos(\theta)+\cos(2\theta)$.
  \begin{align*}
    V(\theta) &= 3+4\cos(\theta)+(2\cos^2(\theta)-1) \\
              &= 2(1+2\cos(\theta)+\cos^2(\theta)) \\
              &= 2(1+\cos(\theta))^2 \\
              &\geq0
  \end{align*}
  Or
  \[ 3F(\sigma)+4\Re(F(\sigma+it))+\Re(F(\sigma+2it)) = \sum_{n\geq1}\frac{a_n V(t\ln n)}{n^\sigma}, \]
  et comme les $a_n$ sont positifs ou nuls, cela implique bien que
  \[ 3F(\sigma)+4\Re(F(\sigma+it))+\Re(F(\sigma+2it))\geq0 \]
\end{proof}

\begin{corollary}\label{cor:zeta-inegalite-1}
  \[ \zeta^3(\sigma)|\zeta(\sigma+it)|^4|\zeta(\sigma+2it)|\geq1\quad(\sigma>1) \]  
\end{corollary}

\begin{proof}
  On applique le théorème à la fonction
  \[ F(s) = \log\zeta(s) = \sum_{n\geq2}\frac{\Lambda(n)}{n^s\ln n}\quad(\sigma>1) \]
  où la 2ème égalité vient de la proposition \ref{prop:log-zeta}. La série de droite est bien une série de Dirichlet à coefficient positifs ou nuls, d'abscisse de convergence 1.
\end{proof}

Nous sommes maintenant en mesure de prouver le résultat suivant, qui va s'avérer très important pour le théorème des nombres premiers.

\begin{theorem}\label{eq:zeta-non-nul-demi-plan-ferme}
  $\zeta$ ne s'annule pas sur le demi-plan fermé $\sigma\geq1$.
\end{theorem}

\begin{proof}
  On sait déjà que $\zeta$ ne s'annule pas sur le demi-plan ouvert $\sigma>1$, voir la proposition \ref{eq:zeta-non-nul-produit-eulerien}.
  \\

  Supposons par l'absurde qu'il existe $t_0$ tel que $\zeta(1+it_0)=0$, on le fixe. On sait que $\zeta$ est dérivable en $1+it_0$. En particulier, en dérivant suivant l'axe des réels,
  \[ \lim_{\sigma\to1^+}\frac{\zeta(\sigma+it_0)-\zeta(1+it_0)}{\sigma-1}\in\mathbb{C}\]
  donc on peut écrire, en appliquant l'hypothèse,
  \[ \zeta(\sigma+it_0)=O_{\sigma\to1^+}(\sigma-1). \]
  De plus, comme $|\zeta(s)|\leq\frac{\sigma}{\sigma-1}$ par la proposition \ref{eq:zeta-majoration-facile}, on peut écrire
  \[ \zeta(\sigma) = O_{\sigma\to1^+}\left(\frac{1}{\sigma-1}\right). \]
  Enfin, par continuité de $\zeta$,
  \[ \zeta(\sigma+2it_0) = O_{\sigma\to1^+}(1) \]
  En regroupant ces trois comportements asymptotiques, il vient :
  \[ \zeta^3(\sigma)|\zeta(\sigma+it)|^4|\zeta(\sigma+2it)| = O_{\sigma\to1^+}(\sigma-1), \]
  en particulier le membre de gauche peut être rendu arbitrairement petit lorsque $\sigma\to1^+$, ce qui contredit l'inégalité \ref{cor:zeta-inegalite-1}.
\end{proof}

\begin{corollary}
  Dans le demi-plan $\sigma\leq0$, la fonction $\zeta$ admet pour seuls zéros les points $-2n, n\geq1$ entier. Ce sont des zéros simples.
\end{corollary}

\begin{proof}
  Analyse. Soit $s$ tel que $\sigma\leq0$. Réécrivons l'équation fonctionelle \ref{thm:equation-fonctionnelle} :
  \[ \zeta(s) = 2^s\pi^{s-1}\sin\left(\frac{\pi s}{2}\right)\Gamma(1-s)\zeta(1-s)\quad (s\in\mathbb{C}\backslash\{0,1\}) \]
  Ainsi, si $s\neq0$, $\sigma\leq0$ et $\zeta(s)=0$, on a forcément $\sin(\frac{\pi s}{2})=0$, et donc $s=-2,-4,...$. Mais il reste à voir si $\zeta(0)$ est nul ou non.

  En multipliant l'équation fonctionnelle par $(1-s)$ et en utilisant $s\Gamma(s)=\Gamma(s+1)$ TODO, on obtient
  \[ (1-s)\zeta(s) = 2^s\pi^{s-1}\sin\left(\frac{\pi s}{2}\right)\Gamma(2-s)\zeta(1-s). \]
  On peut passer à la limite $s\to1$, en remarquant que d'une part $\Gamma(1)=1$, et d'autre part $\zeta(s)\sim\frac{1}{s-1}$ :
  \[ -1 = 2\zeta(0). \]
  Ceci montre que les seuls zéros possibles de $\zeta$ sont les $-2, -4...$.

  Synthèse. On a bien $\zeta(-2)=\zeta(-4)=...=0$
\end{proof}

Les zéros aux entiers négatifs pairs sont appelés les zéros triviaux de la fonction $\zeta$.

%%%%%%%%%%%%%%%%%%%%%%%%%%%%%%%%%%%%%%%%%%%%%%%%%%%%%%%%%%%%%%%%%%%%%%%%%%%%%%%%
\subsection{Répartition globale des zéros}
%%%%%%%%%%%%%%%%%%%%%%%%%%%%%%%%%%%%%%%%%%%%%%%%%%%%%%%%%%%%%%%%%%%%%%%%%%%%%%%%

Développement en produit de Hadamard.

Régions sans zéros qui seront explorées plus en détails dans le chapitre sur le théorème des nombres premiers.

%%%%%%%%%%%%%%%%%%%%%%%%%%%%%%%%%%%%%%%%%%%%%%%%%%%%%%%%%%%%%%%%%%%%%%%%%%%%%%%%
%%%%%%%%%%%%%%%%%%%%%%%%%%%%%%%%%%%%%%%%%%%%%%%%%%%%%%%%%%%%%%%%%%%%%%%%%%%%%%%%
%%%%%%%%%%%%%%%%%%%%%%%%%%%%%%%%%%%%%%%%%%%%%%%%%%%%%%%%%%%%%%%%%%%%%%%%%%%%%%%%
\chapter{Le théorème des nombres premiers}
%%%%%%%%%%%%%%%%%%%%%%%%%%%%%%%%%%%%%%%%%%%%%%%%%%%%%%%%%%%%%%%%%%%%%%%%%%%%%%%%
%%%%%%%%%%%%%%%%%%%%%%%%%%%%%%%%%%%%%%%%%%%%%%%%%%%%%%%%%%%%%%%%%%%%%%%%%%%%%%%%
%%%%%%%%%%%%%%%%%%%%%%%%%%%%%%%%%%%%%%%%%%%%%%%%%%%%%%%%%%%%%%%%%%%%%%%%%%%%%%%%

Nous allons montrer le théorème des nombres premiers dans ce chapitre. Plus précisément, nous allons montrer qu'il découle presqu'immédiatement de l'existence d'une région sans zéro de la fonction $\zeta$ qui déborde sur le demi-plan $\sigma<1$, et que plus cette région sans zéro est grande, mieux est le terme d'erreur dans le théorème des nombres premiers.

%%%%%%%%%%%%%%%%%%%%%%%%%%%%%%%%%%%%%%%%%%%%%%%%%%%%%%%%%%%%%%%%%%%%%%%%%%%%%%%%
%%%%%%%%%%%%%%%%%%%%%%%%%%%%%%%%%%%%%%%%%%%%%%%%%%%%%%%%%%%%%%%%%%%%%%%%%%%%%%%%
\section{Région affaiblie sans zéros}
%%%%%%%%%%%%%%%%%%%%%%%%%%%%%%%%%%%%%%%%%%%%%%%%%%%%%%%%%%%%%%%%%%%%%%%%%%%%%%%%
%%%%%%%%%%%%%%%%%%%%%%%%%%%%%%%%%%%%%%%%%%%%%%%%%%%%%%%%%%%%%%%%%%%%%%%%%%%%%%%%

\begin{theorem}[Théorème des nombres premiers, avec terme d'erreur, version affaiblie]\label{eq:tnp-erreur-1}
  Il existe une constante réelle $c>0$ telle que l'on ait, au voisinage de $+\infty$,
  \begin{equation}
    \psi(x)=x+O(x\mathrm{e}^{-c\ln^{1/10} x})
  \end{equation}
\end{theorem}

L'ingrédient principal pour prouver le théorème des nombres premiers est le fait que $\zeta$ ne s'annule pas sur la droite $\sigma=1$. L'idée intuitive est la suivante : si $\zeta$ a un zéro en $\rho$, alors $\frac{\zeta'}{\zeta}$ a un pôle en $\rho$. Or le produit eulérien nous donne de bonnes informations sur le comportement de $\frac{\zeta'}{\zeta}$ dans le demi-plan $\sigma>1$. En particulier, elle y est bornée. Par conséquent, si $\frac{\zeta'}{\zeta}$  (qui est négative) avait un pôle près de la droite $\sigma=1$, elle devrait y décroître très rapidement.

\begin{theorem}[Région sans zéros, version affaiblie]\label{eq:region-faible-sans-zero}
  Il existe un réel $c>0$ tel que sur la région du plan complexe définie par $|t|\geq2,\,\sigma\geq1-\frac{c}{\ln^9|t|}$, on ait
  \[ \left|\frac{1}{\zeta(s)}\right| = O(\ln^7|t|) \]

  lorsque $t\mapsto+\infty$. En particuler, $\zeta$ ne s'annule pas sur cette région du plan.
\end{theorem}

\begin{proof}
  Par les hypothèses du corollaire \ref{cor:majoration-zeta-et-zeta-prime},
  \begin{itemize}
    \item $\forall c>0$,
    \item $\forall 0<\eta<\frac{c}{\ln|t|}$,
    \item $\forall s=\sigma+it\,\,\mathrm{avec}\,\,\sigma>1-\eta, |t|\geq2$,
  \end{itemize}
  alors on a bien $\sigma > 1-\frac{c}{\ln|t|}$, $|t|\geq2$, et on pose $s_0=1+\eta+it$. Par ce corollaire \ref{cor:majoration-zeta-et-zeta-prime},
  TODO Figure

  \begin{align*}
    |\zeta(s)-\zeta(s_0)|
    = \left|\int_{s_0}^s\zeta'(\omega)\mathrm{d}\omega\right|
    &\leq C_0|s-s_0|\ln^2|t| \\
    &\leq 2C_0\eta\ln^2|t|,
  \end{align*}
  
  où l'intégrale est bien définie car $|t|\geq2$ donc on évite le pôle, et $C_0$ est une constante positive qui vient du $0$ de \ref{cor:majoration-zeta-et-zeta-prime}.

  On utilise maintenant l'inégalité fondamentale \ref{cor:zeta-inegalite-1} due à Mertens:
  \begin{align*}
    \zeta(s_0)
    &\geq \frac{1}{\zeta(1+\eta)^3|\zeta(1+\eta+2it)|} \\
    &\geq \frac{C_1\eta^3}{\ln|t|}
  \end{align*}

  où l'on a utilisé dans la deuxième inégalité la proposition \ref{cor:majoration-zeta-et-zeta-prime}, qui donne la constante positive $C_1$.
  \\

  D'où :
  \begin{align*}
    |\zeta(s)|
    &\geq |\zeta(s_0)| - |\zeta(s)-\zeta(s_0)| \\
    &\geq \frac{C_1^{1/4}\eta^{3/4}}{\ln^{1/4}|t|} - 2C_0\eta\ln^2|t|
  \end{align*}

  Choisissons alors $\eta$ pour que les deux termes se compensent, c'est-à-dire $\eta$ de l'ordre de $\ln^{-9}|t|$.
  \\

  Plus précisément, si l'on choisit :
  \[ \eta:=\frac{c}{\ln^9|t|}, \]

  Alors les hypothèses sont toujours vérifiées, on peut refaire le cheminement ci-dessus, et on a alors
  \[
    |\zeta(s)|
    \geq\frac{C_1^{1/4}c^{3/4}}{\ln^7|t|}-\frac{2C_0c}{\ln^7|t|}
    \geq\frac{C_2}{\ln^7|t|}.
  \]

  Donc $\zeta$ ne s'annule pas sur cette région, et la proposition est démontrée.
\end{proof}

Nous sommes maintenant en mesure de prouver le théorème des nombres premiers sous sa forme \ref{eq:tnp-erreur-1}. L'astuce est d'utiliser la formule de Perron, de briser le segment sur lequel s'effectue l'intégration, et de le remplacer par un contour qui contient le pôle $s=1$ (donc forcément débordant en partie sur le demi-plan $\sigma<1$). Le terme d'erreur vient alors de la valeur de l'intégrale sur les autres parties du contour. Formalisons cela.

\begin{proof}[Démonstration du théorème \ref{eq:tnp-erreur-1}]
  Nous avons vu à la proposition \ref{cor:zeta-sur-zeta-prime-o} 
  \[ \left|\frac{\zeta'(\sigma)}{\zeta(\sigma)}\right| = O_{\sigma\to1^+}\left(\frac{1}{\sigma-1}\right) \]
  Ainsi nous pouvons appliquer la seconde formule de Perron effective \ref{eq:perron-2-effective} avec :
  \begin{itemize}
    \item la série de Dirichlet $F(s)=\sum_{n\geq1}\frac{\Lambda(n)}{n^s}$
    \item $\alpha=1$, $\sigma_a=1$, et on a
      \[
        \sum_{n\geq1}\frac{\Lambda(n)}{n^{-\sigma}}
        = -\frac{\zeta'(\sigma)}{\zeta(\sigma)}
        = O((\sigma-1)^{-1})
        \]
    \item la fonction $B=\ln$ vérifie
    \[ \forall n\geq1,\quad\Lambda(n)\leq B(n) \]
  \end{itemize}
  On obtient, dans ce cas-là, pour $x\geq2$, $s=0$, $T\geq2$, et $\sigma_1=1+\frac{1}{\ln x}$ :
  \[ \psi(x) = \sum_{n\geq1}\Lambda(n) = \frac{1}{2\pi i}\int_{\sigma_1-iT}^{\sigma_1+iT}\left(-\frac{\zeta'(s)}{\zeta(s)}\right)\frac{x^s}{s}\mathrm{d}s\,+\,O\left(\frac{x\ln^2 x}{T}\right). \]
  Nous allons utiliser le théorème des résidus pour estimer l'intégrale, et choisir $T$ à la fin convenablement pour compenser les termes d'erreur.

  En fixant $c$ donné par le théorème \ref{eq:region-faible-sans-zero}, posons $\delta=\frac{c}{\ln^9(T+2)}$, $\sigma_0=1-\delta$, et rappelons que nous venons de poser $\sigma_1=1+\ln x$. Alors le rectangle de sommets $(\sigma_0,\pm iT)$ et $(\sigma_1,\pm iT)$ est contenu dans cette région sans zéros. En notant $f(s)=\left(-\frac{\zeta'(s)}{\zeta(s)}\right)\frac{x^s}{s}$ l'intégrande, on a alors que $s=1$ est l'unique pôle de $f$ dans ce rectangle. Le théorème des résidus nous fournit alors (le terme à l'intérieur de l'intégrale, $f(s)\mathrm{d}s$, est omis pour clarté) :
  \[ \psi(x) = \mathrm{Res}(f,1) + \frac{1}{2\pi i}\left(
    \int_{\sigma_0-iT}^{\sigma_0+iT}
    +\int_{\sigma_0+iT}^{\sigma_1+iT}
    -\int_{\sigma_0-iT}^{\sigma_1-iT}
    \right)
    +O\left(\frac{x\ln^2x}{T}\right). \]

  TODO IMAGE

  Nous pouvons facilement vérifier que $\mathrm{Res}(f,1)=x$, car $\zeta$ a un pôle simple en $s=1$, donc
  \begin{equation}\label{eq:tnp-faible-3-integrales}
    \psi(x) = x +\frac{1}{2\pi i}\left(
      \int_{\sigma_0-iT}^{\sigma_0+iT}
      +\int_{\sigma_0+iT}^{\sigma_1+iT}
      -\int_{\sigma_0-iT}^{\sigma_1-iT}
    \right)
    +O\left(\frac{x\ln^2x}{T}\right)
  \end{equation}

  On voit déjà le terme $x$ de l'énoncé apparaître, essayons donc d'estimer les trois intégrales. Commençons par celle du milieu, qui a lieu sur le petit segment horizontal du haut. Notons $I_h$ ce segment, et on a la majoration grossière :
  \begin{align*}
    |f(s)| = \left|\left(-\frac{\zeta'(s)}{\zeta(s)}\right)\frac{x^s}{s}\right| & \leq\frac{x^{\sigma_1}}{T}\max_{I_h}|\zeta'|\max_{I_h}\left|\frac{1}{\zeta}\right| \\
    & \leq\frac{x}{T}\max_{I_h}|\zeta'|\max_{I_h}\left|\frac{1}{\zeta}\right|.
  \end{align*}
  (Astuce : l'inégalité $|s|\geq T$ se voit mieux géométriquement.)

  Mais on a déjà vu :
  \begin{itemize}
    \item $\max|\zeta'| = O(\ln^2 T)$ dans la proposition \ref{cor:majoration-zeta-et-zeta-prime},
    \item $\max|\frac{1}{\zeta}| = O(\ln^7 T)$ dans le théorème juste au-dessus \ref{eq:region-faible-sans-zero}.
  \end{itemize}

  Et donc
  \[ |f(s)| = O\left(\frac{x}{T}\ln^9T\right) \]

  Comme la longueur du segement est $\sigma_1-\sigma_0=1/\ln x+\delta=O(1)$, on a que
  \begin{equation}\label{eq:tnp-faible-3-integrales-part1}
    \int_{\sigma_0+iT}^{\sigma_1+iT}f(s)\mathrm{d}s = O\left(\frac{x}{T}\ln^9T\right)
  \end{equation}

  Par symétrie par rapport à l'axe des abscisses, la troisième intégrale sur le segment de bas $I_b$ dans \ref{eq:tnp-faible-3-integrales} est également $O\left(\frac{x}{T}\ln^9T\right)$.

  Reste à évaluer la première intégrale sur le segment vertical de gauche $I_g$. Pareil, majorons grossièrement :
  \begin{align*}
    |f(s)| = \left|\left(-\frac{\zeta'(s)}{\zeta(s)}\right)\frac{x^s}{s}\right| & \leq\frac{x^{\sigma_0}}{|s|}\max_{I_g}|\zeta'|\max_{I_g}\left|\frac{1}{\zeta}\right| \\
    & = O\left(\frac{x^{\sigma_0}}{|s|}\max_{I_g}|\zeta'|\ln^7 T\right)
  \end{align*}

  Ici, il faut faire attention car ce segment est assez proche du pôle $s=1$, donc $\max_{I_g}|\zeta'|$ peut devenir assez importante. Nous utilisons la proposition TODO pour avoir la majoration
  % https://web.stanford.edu/~tonyfeng/riemann_zeta.pdf
  \begin{equation}\label{eq:tnp-faible-3-integrales-part2}
    \max_{I_g}|\zeta'|=O(\ln^{18}T)
  \end{equation}

  En combinant \ref{eq:tnp-faible-3-integrales-part1} et \ref{eq:tnp-faible-3-integrales-part2} et en les injectant dans \ref{eq:tnp-faible-3-integrales}, on obtient
  \[ \psi(x) = x
  + O\left(x^{\sigma_0}\ln^{25}x\int_{-T}^T\frac{1}{1+|t|}\mathrm{d}t\right)
  + O\left(\frac{x}{T}\ln^9T\right)
  + O\left(\frac{x\ln^2x}{T}\right)
  \]

  L'intégrale est $O(\ln T)$, et le dernier terme peut être absorbé dans $O\left(\frac{x}{T}\ln^9T\right)$, donc
  \[ \psi(x) = x + O(x^{\sigma_0}\ln^{26}x) + O\left(\frac{x}{T}\ln^9T\right) \]

  En majorant encore plus grossièrement, c'est-à-dire $O(\ln^9 T) = O(\ln^26 T)$, on peut factoriser
  \begin{align*}
    \psi(x) &= x + O\left(x\ln^{26} x\left(x^{-\delta}+\frac{1}{T}\right)\right) \\
    &= x + O\left(x\ln^{26} x\left(\mathrm{e}^{-\frac{c\ln x}{\ln^9(T+2)}}+\mathrm{e}^{-\ln T}\right)\right)
  \end{align*}

  Il faut noter que les deux termes en exponentiel varient en sens opposés lorsque $T$ augmente. On peut les rendre égaux en choisissant $T=\exp(\ln^{1/10}x)$, auquel cas on a :
  \[ \psi(x) = x + O(x\ln^{26} x\mathrm{e}^{-c\ln^{1/10}x}). \]

  En choisissant $c$ assez grand TODO, on obtient Enfin
  \[ \psi(x) = x + O(x\mathrm{e}^{-c\ln^{1/10}x}). \]
\end{proof}

%%%%%%%%%%%%%%%%%%%%%%%%%%%%%%%%%%%%%%%%%%%%%%%%%%%%%%%%%%%%%%%%%%%%%%%%%%%%%%%%
%%%%%%%%%%%%%%%%%%%%%%%%%%%%%%%%%%%%%%%%%%%%%%%%%%%%%%%%%%%%%%%%%%%%%%%%%%%%%%%%
\section{Région classique sans zéros}
%%%%%%%%%%%%%%%%%%%%%%%%%%%%%%%%%%%%%%%%%%%%%%%%%%%%%%%%%%%%%%%%%%%%%%%%%%%%%%%%
%%%%%%%%%%%%%%%%%%%%%%%%%%%%%%%%%%%%%%%%%%%%%%%%%%%%%%%%%%%%%%%%%%%%%%%%%%%%%%%%

Le terme d'erreur $O(x\mathrm{e}^{-c\ln^{1/10}x})$ dans \ref{eq:tnp-erreur-1} n'est pas terrible, on peut peut-être l'améliorer. En diagnostiquant la preuve ci-dessus, on remarque que ce terme d'erreur vient en très grande partie de l'intégrale sur le segment vertical de gauche. En fait, plus on arrive à décaler ce segment vers la gauche, plus l'intégrale sera petite.

Mais pour ce faire, il nous faut trouver une région sans zéros plus grande que celle du théorème \ref{eq:region-faible-sans-zero}. Le but de cette section est donc de démontrer le théorème suivante :

\begin{theorem}[Région classique sans zéro]\label{eq:region-classique-sans-zero}
  Il existe un réel $c>0$ tel que la fonction $\zeta$ ne s'annule pas sur la région du plan définie par
  \[ \sigma\geq1-\frac{c}{\ln(2+|t|)} \]
\end{theorem}

\begin{proof}
  Nous partons de la propositions \ref{cor:zeta-zeta-prime-van-mangoldt}, qui nous fournit la série de Dirichlet
  \[ -\frac{\zeta'(s)}{\zeta(s)} = \sum_{n\geq1}\frac{\Lambda(n)}{n^s}. \]

  Son abscisse de convergence est également $1$, et on a d'après la formule \ref{thm:mertens-positif} due à Mertens, $\forall\sigma>1, \forall\gamma\in\mathbb{R}$

 \begin{equation}\label{eq:mertens-zeta-zeta-prime}
  - 3\frac{\zeta'(\sigma)}{\zeta(\sigma)}
  - 4\Re\frac{\zeta'(\sigma+i\gamma)}{\zeta(\sigma+i\gamma)}
  - \Re\frac{\zeta'(\sigma+2i\gamma)}{\zeta(\sigma+2i\gamma)}
  \geq0.
 \end{equation}

  Nous allons majorer les 3 termes ci-dessus lorsque $\gamma$ est l'ordonnée d'un zéro $\rho=\beta+i\gamma$ de $\zeta$.

  Premièrement,
  \[ -\frac{\zeta'(\sigma)}{\zeta(\sigma)} = \frac{1}{\sigma-1} + O(1). \]

  Deuxièmement, en dérivant logarithmitiquement la formule du produit de Hadamard \ref{thm:produit-hadamard} de $\zeta$, on a:

  \begin{align*}
    -\frac{\zeta'(s)}{\zeta(s)}
    &= -b + \frac{1}{s-1}
    + \frac{\Gamma'(\frac{s}{2}+1)}{2\Gamma(\frac{s}{2}+1)}
    - \sum_{\rho}\left(\frac{-1}{1-\frac{s}{\rho}} + \frac{1}{\rho}\right) \\
    &= -b + \frac{1}{s-1}
    + \frac{\Gamma'(\frac{s}{2}+1)}{2\Gamma(\frac{s}{2}+1)}
    - \sum_{\rho}\left(\frac{1}{\rho} + \frac{1}{s-\rho} \right)
  \end{align*}

  En utlisant la formule de Stirling complexe \ref{thm:stirling-complexe} POURQUOI, on a
  \[ -\Re\frac{\zeta'(s)}{\zeta(s)} = O(\ln(|t|))\quad(\sigma>1, |t|\geq2) \]

  et aussi, en tenant compte de la contribution du zéro $\beta+i\gamma$, POURQUOI
  \[ aaa \]

  En reportant dans \ref{eq:mertens-zeta-zeta-prime},
  \[
    \frac{-3}{\sigma - 1} -3 M_1(\sigma)
    - \frac{4}{\sigma-\beta} + 4M_2(|\gamma|)\ln|\gamma|
    + M_3(|\gamma|)\ln|\gamma|
    \geq0,
  \]

  En notant $c_1$, on a alors, pour $\gamma\geq2$
  \[
    \frac{3}{\sigma-1}-\frac{4}{\sigma-\beta}\geq c_1\ln|\gamma|
  \]

  d'où
  \[
    1-\beta\geq\frac{1-c_1(\sigma-1)\ln|\gamma|}{(3/(\sigma-1))+c_1\ln|\gamma|}
  \]

  TODO

\end{proof}

\begin{theorem}[Théorème des nombres premiers avec terme d'erreur, de la Vallée Poussin, 1900]\label{eq:tnp-vallee-poussin}
  Il existe une constante réelle $c>0$ telle que l'on ait, au voisinage de $+\infty$,
  \begin{equation}
    \psi(x)=x+O(x\mathrm{e}^{-c\sqrt{\ln x}})
  \end{equation}
\end{theorem}

\begin{proof}
  La preuve est exactement la même que celle du théorème \ref{eq:tnp-erreur-1}, sauf qu'on utilise la nouvelle région sans zéros que l'on vient de définir plus haut, et le rectangle est de sommets $(\sigma_0,\pm iT)$, $(\sigma_1,\pm iT)$, avec 
  \begin{itemize}
    \item $\sigma_0=1-\frac{c}{\ln(2+T)}$
    \item $\sigma_1=1+1/\ln x$
  \end{itemize}
  On réécrit :
  \begin{equation}\label{eq:tnp-3-integrales}
    \psi(x) = x +\frac{1}{2\pi i}\left(
      \int_{\sigma_0-iT}^{\sigma_0+iT}
      +\int_{\sigma_0+iT}^{\sigma_1+iT}
      -\int_{\sigma_0-iT}^{\sigma_1-iT}
    \right)
    +O\left(\frac{x\ln^2x}{T}\right)
  \end{equation}
  Mais cette fois, sur les deux segments horizontaux $I_h$ et $I_b$ qui sont $[\sigma_0\pm iT,\sigma_1\pm iT]$, on a, utilisant la proposition TODO
  \begin{align}
    |f(s)| = \left|\left(-\frac{\zeta'(s)}{\zeta(s)}\right)\frac{x^s}{s}\right| & \leq\frac{x^{\sigma_1}}{T}\max_{I_h}\left|\frac{\zeta'}{\zeta}\right| \\
    & = O\left(\frac{x^{\sigma_1}\ln(T+2)}{T}\right) \\
    & = O\left(\frac{x\ln x}{T}\right) \label{eq:tnp-grand-O}
  \end{align}
  où la dernière égalité vient du fait que l'on n'oubliera pas, à la fin de la preuve, de choisir $T$ tel que $T=O(x)$ (et bien sûr grand, pour que le $2$ parte aussi).

  Sur le segment vertical de gauche $I_g=[\sigma_0-iT, \sigma_0+iT]$, on a
  \begin{align*}
    |f(s)| = \left|\left(-\frac{\zeta'(s)}{\zeta(s)}\right)\frac{x^s}{s}\right| & \leq\frac{x^{\sigma_0}}{|s|}\max_{I_g}\left|\frac{\zeta'}{\zeta}\right| \\
    & = O\left(\frac{x^{\sigma_0}\ln(T+2)}{|s|}\right) \\
    & = O\left(\frac{x^{1-c/\ln(T+2)}\ln x}{|s|}\right)
  \end{align*}
  où la deuxième ligne est justifiée par TODO.

  En réassemblant tout, on obtient
  \begin{align*}
    \psi(x) &= x + O\left(x^{1-\frac{c}{\ln(T+2)}}\ln x\int_{-T}^T\frac{\mathrm{d}t}{1+|t|}\right) + O\left(\frac{x\ln^2x}{T}\right) \\
    &= x + O\left(x\ln^2 x\left(\mathrm{e}^{-\frac{c\ln x}{\ln(T+2)}}+\mathrm{e}^{-\ln T}\right)\right)
  \end{align*}

  En choisissant alors $T:=\mathrm{e}^{\sqrt{c\ln x}}$, on vérifie que $T=O(x)$ pour que \ref{eq:tnp-grand-O} soit justifiée, et on a aussi dans ce cas
  \[ \psi(x)=x+O(x\mathrm{e}^{-c\sqrt{\ln x}}) \]
\end{proof}

%%%%%%%%%%%%%%%%%%%%%%%%%%%%%%%%%%%%%%%%%%%%%%%%%%%%%%%%%%%%%%%%%%%%%%%%%%%%%%%%
%%%%%%%%%%%%%%%%%%%%%%%%%%%%%%%%%%%%%%%%%%%%%%%%%%%%%%%%%%%%%%%%%%%%%%%%%%%%%%%%
\section{Enoncé avec $\pi(x)$}
%%%%%%%%%%%%%%%%%%%%%%%%%%%%%%%%%%%%%%%%%%%%%%%%%%%%%%%%%%%%%%%%%%%%%%%%%%%%%%%%
%%%%%%%%%%%%%%%%%%%%%%%%%%%%%%%%%%%%%%%%%%%%%%%%%%%%%%%%%%%%%%%%%%%%%%%%%%%%%%%%

Nous avons vu, dans les deux section précédentes, deux théorèmes que nous avons appelés "théorème des nombres premiers", mais qui ne faisaient intervenir que $\psi(x)$, sans aucun lien apparent avec $\pi(x)$. Le théorème des nombres premiers s'appelle ainsi parce qu'il a quelque chose à dire sur la distribution des nombres premiers :

\begin{theorem}[Théorème des nombres premiers]\label{eq:tnp-pi-x}
  Il existe une constante réelle $c>0$ telle que l'on ait, au voisinage de $+\infty$,
  \[ \pi(x) = \int_2^x\frac{\mathrm{d}t}{\ln t} + O(x\mathrm{e}^{-c\sqrt{\ln x}}) \]
\end{theorem}

Cette intégrale a en réalité un nom :

\begin{definition}[Logarithme intégral] On définit le logarithme intégral, noté $\mathrm{li}$, par
  \[ \mathrm{li}(x) = \int_0^x\frac{\mathrm{d}t}{\ln t}. \]

  On définit également la fonction d'écart logarithmique intégrale :
  \[
    \mathrm{Li}(x)
    = \int_2^x\frac{\mathrm{d}t}{\ln t}
    = \mathrm{li}(x) - \mathrm{li}(2)
  \]
\end{definition}

Remarquons que $t\mapsto\frac{1}{\ln t}$ n'est pas définie pour $t=1$, et la définition de $\mathrm{li}$ doit être interprétée en valeur principale de Cauchy :
\[
  \mathrm{li}(x)
  = \lim_{\epsilon\mapsto0}\left(
      \int_0^{1-\epsilon}\frac{\mathrm{d}t}{\ln t}
      + \int_{1-\epsilon}^\infty\frac{\mathrm{d}t}{\ln t}
    \right).
  \]

\begin{proof}[Démonstration du théorème \ref{eq:tnp-pi-x}]
  TODO
  % https://etd.ohiolink.edu/!etd.send_file?accession=kent1550224160190008&disposition=inline p42
\end{proof}

On en déduit également la forme suivante, plus faible, mais plus répandue :

\begin{theorem}[Théorème des nombres premiers, version affaiblie]\label{eq:tnp-equivalence}
  \[ \pi(x)\sim\int_2^x\frac{\mathrm{d}t}{\ln t} \]
\end{theorem}

\begin{proof}
  Il suffit juste de voir que (TODO Pourquoi?)
  \[ x\mathrm{e}^{-c\sqrt{\ln x}} = o(\mathrm{li}(x)). \]
\end{proof}

\begin{proposition}
  \[ \mathrm{li}(x) = \frac{x}{\ln x} + \frac{x}{\ln^2 x} + O\left(\frac{x}{\ln^3 x} \right) \]
\end{proposition}

\begin{proof}
  Cela se fait facilement par intégration par parties.
\end{proof}

On a donc $\pi(x)\sim\mathrm{li}(x)\sim\frac{x}{\ln x}$, mais $\mathrm{li}$ donne une meilleure approximation :

TODO GRAPHE

%%%%%%%%%%%%%%%%%%%%%%%%%%%%%%%%%%%%%%%%%%%%%%%%%%%%%%%%%%%%%%%%%%%%%%%%%%%%%%%%
%%%%%%%%%%%%%%%%%%%%%%%%%%%%%%%%%%%%%%%%%%%%%%%%%%%%%%%%%%%%%%%%%%%%%%%%%%%%%%%%
\section{L'hypothèse de Riemann}
%%%%%%%%%%%%%%%%%%%%%%%%%%%%%%%%%%%%%%%%%%%%%%%%%%%%%%%%%%%%%%%%%%%%%%%%%%%%%%%%
%%%%%%%%%%%%%%%%%%%%%%%%%%%%%%%%%%%%%%%%%%%%%%%%%%%%%%%%%%%%%%%%%%%%%%%%%%%%%%%%

TODO

\begin{theorem}
  L'hypothèse de Riemann équivaut à
  \[ \forall\epsilon>0,\quad\psi(x)=x+O_\epsilon(x^{1/2+\epsilon}) \]
\end{theorem}

%%%%%%%%%%%%%%%%%%%%%%%%%%%%%%%%%%%%%%%%%%%%%%%%%%%%%%%%%%%%%%%%%%%%%%%%%%%%%%%%
%%%%%%%%%%%%%%%%%%%%%%%%%%%%%%%%%%%%%%%%%%%%%%%%%%%%%%%%%%%%%%%%%%%%%%%%%%%%%%%%
\section{Formule explicite pour $\psi$}
%%%%%%%%%%%%%%%%%%%%%%%%%%%%%%%%%%%%%%%%%%%%%%%%%%%%%%%%%%%%%%%%%%%%%%%%%%%%%%%%
%%%%%%%%%%%%%%%%%%%%%%%%%%%%%%%%%%%%%%%%%%%%%%%%%%%%%%%%%%%%%%%%%%%%%%%%%%%%%%%%

% Random citation \cite{tenenbaum} embeddeed in text.
% Random citation \cite{standord} embeddeed in text.


% \bibliographystyle{ieeetr}
% \bibliography{main}{}

\end{document}
